\documentclass[journal,12pt,twocolumn]{IEEEtran}

\usepackage{setspace}
\usepackage{amssymb}
\usepackage{amsthm}
\usepackage{mathrsfs}
\usepackage{enumitem}
\usepackage{mathtools}
\usepackage{float}
\usepackage{caption}
\usepackage{graphicx}
\usepackage{babel}
\usepackage{amsmath}

\usepackage[breaklinks=true]{hyperref}

\usepackage{listings}
    \usepackage{color}                                            %%
    \usepackage{array}                                            %%
    \usepackage{longtable}                                        %%
    \usepackage{calc}                                             %%
    \usepackage{multirow}                                         %%
    \usepackage{hhline}                                           %%
    \usepackage{ifthen}                                           %%
    \usepackage{lscape}     
    \usepackage{amsmath}
       
\lstset{
%language=C,
frame=single, 
breaklines=true,
columns=fullflexible
}
\def\inputGnumericTable{}
\bibliographystyle{IEEEtran}

\newcommand{\question}{\noindent \textbf{Question: }}
\newcommand{\solution}{\noindent \textbf{Solution: }}
\providecommand{\cbrak}[1]{\ensuremath{\left\{#1\right\}}}
\providecommand{\brak}[1]{\ensuremath{\left(#1\right)}}
\providecommand{\pr}[1]{\ensuremath{\Pr\left(#1\right)}}
\newcommand{\mydet}[1]{\ensuremath{\begin{vmatrix}#1\end{vmatrix}}}
\newcommand{\myvec}[1]{\ensuremath{\begin{pmatrix}#1\end{pmatrix}}}
\newcommand*{\permcomb}[4][0mu]{{{}^{#3}\mkern#1#2_{#4}}}
\newcommand*{\perm}[1][-3mu]{\permcomb[#1]{P}}
\newcommand*{\comb}[1][-1mu]{\permcomb[#1]{C}}

\title{Assignment 9}
\author{AKHILA, CS21BTECH11031}

\begin{document}
% make the title area
\maketitle
\question\\
Suppose the Conditional distribution of X given Y = n is binomial with parameters n and 
$p_1$. Further, Y is a binomial random variable with parameters M and $p_2$. Show that the 
distribution of X is also binomial. Find its parameters.\\ 

\solution\\
If X has the binomial distribution B(m, p) given by 
\begin{align}
    p_n = \pr{X=n} = \myvec{m \\ n}p^nq^{m-n},0\leq n\leq m
\end{align}
then\\ 
Moment generating function
\begin{align}
    \Gamma(z)=E\cbrak{z^{X}} = \sum_{n=0}^m  \myvec{m \\ n}p^nq^{m-n}z^n=(pz+q)^m 
\end{align}
Given that the Conditional distribution of X given Y = n is binomial with parameters n and 
$p_1$.
\begin{multline}
\pr{X=k|Y=n}=\myvec{n \\ k}p_1^kq_1^{n-k} ,k=0,1,2,\dots n\\
E[z^{X}|Y=n]=\sum_{k=0}^n z^{k}\pr{X=k|Y=n}\\=\brak{p_1 z+q_1}^n
\end{multline}
Also,We have
\begin{align}
\Gamma(z)=E\cbrak{z^{X}}=E\cbrak{E[z^{X}|Y=n]}\\
=\sum_{n=0}^M E[z^{X}|Y=n]\pr{Y=n}\\
=\sum_{n=0}^M \brak{p_1 z+q_1}^n \myvec{M \\ n}p_2^nq_2^{M-n}\\
=\sum_{n=0}^M\myvec{M \\ n}[p_2\brak{p_1 z+q_1}]^n q_2^{M-n}\\
=\brak{{p_1p_2 z+p_2q_1+q_2}}^M
\end{align}
But
\begin{align}
1-p_1p_2 = 1-\brak{1-q_1}\brak{1-q_2} = q_1p_2+q_2
\end{align}
Hence
\begin{align}
\Gamma(z)=\brak{pz+q}^M
\end{align}
Where $p=p_1p_2$.\\ Therefore,
X $\sim$ Binomial$\brak{M,p_1p_2}$ \\i.e,
The distribution of X is also binomial with parameters M and $p_1p_2$.
\end{document}